\documentclass{emulateapj}
\usepackage{amsmath} 
\usepackage{apjfonts} 
\usepackage{amssymb} 
\usepackage{xspace}
\usepackage{hyperref} 
\usepackage{natbib} 
\usepackage{multirow}
\usepackage{graphicx}
\usepackage{epstopdf}
\usepackage{epsf}
\usepackage{subfigure}
\bibliographystyle{apj_ads}
\usepackage[usenames,dvipsnames]{xcolor}
\newcommand{\comment}[1]{{\color{red} #1}}
\newcommand{\change}[1]{{\color{blue} #1}}
\newcommand{\todo}[1]{{\bf\color{blue} #1}}
%%%%%%%%%% User defined symbol %%%%%%
\def\gsim{\gtrsim}
\def\lsim{\lesssim}    
\def\Msun{M_\odot}
%%%%%%%%%%%%%%%%%%%%%%%%%%%%%%%%%%%%%

\begin{document} 
\shorttitle{LensFinder}
\shortauthors{Avestruz, Li, Lightman}
\submitted{The Astrophysical Journal} 
\slugcomment{The Astrophysical Journal, submitted}

\title{LensFinder: A Strong Lensing Identification Pipeline}

\author{Camille Avestruz\altaffilmark{1-3}\thanks{E-mail:
    avestruz@uchicago.edu}, Nan Li\altaffilmark{2,3,4}, and Matthew Lightman\altaffilmark{5} }

\affil{
$^1${Enrico Fermi Institute, The University of Chicago, Chicago, IL 60637 U.S.A.}\\
$^2${Kavli Institute for Cosmological Physics, The University of Chicago, Chicago, IL 60637 U.S.A.}\\
$^3${Department of Astronomy \& Astrophysics, The University of Chicago, Chicago, IL 60637 U.S.A.};
  \href{mailto:avestruz@uchicago.edu}{avestruz@uchicago.edu}\\
}

\keywords{cosmology: theory --- galaxies: clusters: general ---
  galaxies: clusters --- methods : numerical }
  
%-------------------------------------------------%
\begin{abstract} 
  Giant arcs trace... cosmology... mass distribution.  Advent of DES,
  LSST, etc. need an efficient way to identify strong lens images.  We
  present LensFinder, open source image processing and deep learning.
  Tests with mock images from ... ...  Image processing improves by....
\end{abstract}
%-------------------------------------------------%

%-------------------------------------------------%
\section{Introduction}
%-------------------------------------------------%

% Gravitational lensing as a geometric (?) test of cosmology.  Also
% probes underlying matter distribution in galaxy clusters, sensitive
% to things like recent accretion and environment.

% Arc identification can be done visually (e.g. that zooniverse
% thing).  But, in the advent of optical surveys, DES and LSST, we
% will be limited by our ability to automatically process the large
% amounts of data.

% Arc finding must optimize for completeness and purity.  Completeness
% (no false negatives), purity (minimal false positives).  Purity can
% be double checked in a smaller subsample of the data visually.  And,
% spectroscopic follow up is expensive.

% Machine learning to automate.  Previous works (StrongML),
% More+... Here, we focus on the improvement from preprocessing the
% images and describe the pipeline, which is publicly available.

% Image preprocessing.... same technology to validate watermarks


Our paper is organized as follows.  In Section~\ref{sec:methods} we
briefly describe ...  We present our results in
Section~\ref{sec:results}, and our summary and discussions in
Section~\ref{sec:conclusions}.

%-------------------------------------------------%
\section{Methodology}
\label{sec:methods}
%-------------------------------------------------%

%-------------------------------------------------%
\subsection{Image Processing Pipeline}
%-------------------------------------------------%

%-------------------------------------------------%
\subsection{Deep Learning Algorithm}
%-------------------------------------------------%

%-------------------------------------------------%
\subsection{Mock Test Images}
%-------------------------------------------------%


%-------------------------------------------------%
\section{Results}
\label{sec:results}
%-------------------------------------------------%


%-------------------------------------------------%
\section{Summary and Discussions}
\label{sec:conclusions}
%-------------------------------------------------%

We summarize key points below:

%%%%%%%%%%%%%%%%%%%%%%%%%%
\begin{itemize}

\item 
    
\end{itemize}
%%%%%%%%%%%%%%%%%%%%%%%%%%

These results indicate that...

\acknowledgments CA acknowledges support from the Kavli Institute of
Cosmological Physics, the Enrico Fermi Institute at the University of
Chicago, and the University of Chicago Provost's Office.
\lastpagefootnotes


\bibliography{ms}

\end{document}  

