\documentclass{emulateapj}
\usepackage{amsmath} 
\usepackage{apjfonts} 
\usepackage{amssymb} 
\usepackage{xspace}
\usepackage{hyperref} 
\usepackage{natbib} 
\usepackage{multirow}
\usepackage{graphicx}
\usepackage{epstopdf}
\usepackage{epsf}
\usepackage{subfigure}
\bibliographystyle{apj_ads}
\usepackage[usenames,dvipsnames]{xcolor}
\newcommand{\comment}[1]{{\color{red} #1}}
\newcommand{\wording}[1]{{\it\color{purple} #1}}
\newcommand{\todo}[1]{{\bf\color{blue} #1}}
%%%%%%%%%% User defined symbol %%%%%%
\def\gsim{\gtrsim}
\def\lsim{\lesssim}    
\def\Msun{M_\odot}
%%%%%%%%%%%%%%%%%%%%%%%%%%%%%%%%%%%%%

\begin{document} 
\shorttitle{Automated Lensing Learner}
\shortauthors{Avestruz, et al.}
\submitted{The Astrophysical Journal} 
\slugcomment{The Astrophysical Journal, submitted}

\title{Automated Lensing Learner - I: An automated Strong Lensing
  Identification Pipeline}

\author{Camille Avestruz\altaffilmark{1-3,$\dagger$}\thanks{E-mail:
    avestruz@uchicago.edu}, Nan Li\altaffilmark{2-4}, Matthew Lightman\altaffilmark{5}, Thomas E.~Collett\altaffilmark{6}}

\affil{
$^1${Enrico Fermi Institute, The University of Chicago, Chicago, IL 60637 U.S.A.}\\
$^2${Kavli Institute for Cosmological Physics, The University of Chicago, Chicago, IL 60637 U.S.A.}\\
$^3${Department of Astronomy \& Astrophysics, The University of Chicago, Chicago, IL 60637 U.S.A.};\\
$^4${High Energy Physics Division, Argonne National Laboratory,
Lemont, IL 60439};\\
$^5${Digital Intelligence, JPMorgan Chase, Chicago, IL 60603 U.S.A.};\\
$^6${Institute of Cosmology and Gravitation, University of Portsmouth, Burnaby Rd, Portsmouth, PO1 3FX, UK};\\
$^\dagger${Provost's Postdoctoral Scholar at the University of Chicago}\\
  \href{mailto:avestruz@uchicago.edu}{avestruz@uchicago.edu}\\
}

\keywords{gravitational lensing --- methods : numerical --- methods: data analysis --- methods: statistical --- galaxies: elliptical --- surveys  }
  
%-------------------------------------------------%
\begin{abstract} 
Gravitational lensing offers a direct probe of the underlying mass
distribution of lensing systems, a window to the high redshift
universe, and a geometric probe of cosmological models.  The advent of
large scale surveys such as the Large Synoptic Sky Telescope and
Euclid has prompted a need for automatic and efficient identification
of strong lensing systems.  We present {\em (ALL) Automated Lensing
  Learner}, a strong lensing identification pipeline that will be
publicly released as open source software.  In this first application,
we employ a fast feature extraction method, Histogram of Oriented
Gradients (HOG), to capture edge patterns that are characteristic of
strong gravitational arcs in galaxy-galaxy lensing.  We use logistic
regression to train a supervised classifier model on the HOG of HST-
and LSST-like images.  We use an area under the curve (AUC) of a
Receiver Operating Characteristic curve to assess model performance;
$AUC=1.0$ is an ideal trained classifier, and $AUC=0.5$ does not
provide any classification information.  Our best performing models on
a training set of 10,000 lens containing images and 10,000 non-lens
containing images exhibit an AUC of 0.975 for an HST-like sample.
However, for one orbit of LSST, our model only reaches an AUC of
0.625.  For 10-year mock LSST observations, the AUC improves to 0.809.
Model performance appears to continually improve with the size of the
training set.  Models trained on fewer images perform better in
absence of the lens galaxy, which can decrease the false positive rate
by blending with light from a lensed galaxy image near the central
lensing galaxy.  However, with larger training data sets, information
from the lensing galaxy improves model performance.  \wording{Our
  results show efficient and effective methods for automated strong
  lensing identification with scalable linear methods that are easy to
  parallelize with existing open source tools and are reproducible on
  personal laptop computers.}
\end{abstract}
%-------------------------------------------------%

%-------------------------------------------------%
\section{Introduction}
%-------------------------------------------------%

% Gravitational lensing as a geometric test of cosmology.  Also
% probes underlying matter distribution in galaxy clusters, sensitive
% to things like recent accretion and environment.
Gravitational lensing occurs when intermediate fluctuations in the
matter density field deflect light from background sources
\citep[see][for a review]{kneibandnatarajan_11}.  Strong gravitational
lensing can manifest as visible giant arcs magnifying high redshift
galaxies \citep{lyndsandpetrosian_86,gladders_etal03}, multiply imaged
quasars \citep{walsh_etal79}, multiply imaged galaxies
\citep{sharon_etal05}, and arclets \citep{bezecourt_etal98}.  Lensing
signatures probe the underlying dark matter distribution of the lens
\citep{warrenanddye_03}, high redshift galaxy formation
\citep{allam_etal07}, and provide a geometric test of cosmology via
comparison of predicted arc abundances and observed abundances
\citep{kochanek_96,chae_03,linder_04} and time-delay signals from
multiply imaged quasars\citep{xli_etal12,suyu_etal14,suyu_etal16}.

% Application of constraining mass profiles - large numbers needed.
The application of strong gravitational lensing to constrain the mass
distribution of strong lenses, such as early-type galaxies (ETGs),
necessitates large samples of galaxy-galaxy lensing systems.
\citet{miraldaescudeandlehar_92} first suggested that massive
ellipticals would be likely frequent strong lensing sources in optical
surveys.  These systems exhibit a background source galaxy that the
lens galaxy deflects into a partial of full arc shaped Einstein ring.
The strong lensing signature directly probes the underlying matter.
The identification of such systems in upcoming surveys is the first
step in constraining the mass-to-light ratio for a large number of
objects in this mass range.

% Image identification works: visual, ``robot'' identification -
% ArcFinder, RingFinder..., and citizen science (space warps).  Many
% of these rely on some explicit aspect of the image's morphology
% (give example), filtering, ... ....  Specifics of the methodology
% how the instrument affects the strong lensing signal, and the
% objects of interest.
Over the last decade, infrastructure for both large scale visual and
automated image classification emerged. By nature, the human eye is
one of the best discriminators for image classification.  Visual arc
identification has been effective through the use of citizen science
platforms.  {\em SpaceWarps} is an example of citizen science based
image classification of strong lensing systems in Canada France Hawaii
Telescope Science (CFHTLS) telescope observations
\citep{marshall_etal16,more_etal16}.  These platforms are quite
successful for a dataset like CFHTLS, classifying \todo{XX in XX
  time}\citep{}.  However, future datasets like Euclid
\citep{oguriandmarshall_10} and LSST \cite{} will be \todo{XX} times
this size, and {\em Euclid} is estimated to find tens of thousands
galaxy-scale strong lenses \citep{pawase_etal12}.  The volume of
upcoming data challenges the scalability of a citizen science
approach.

% Transition to automated identification
Recent efforts have focused on the development of automated methods
with performance comparable to or better than humans. {\em SpaceWarps}
is a part of the Zooniverse Project \citep{marshall_etal16}, which
also includes {\em Galaxy Zoo}, the citizen science based image
classification of galaxy types \citep{lintott_etal08}.  Galaxy
classification is an early successful example of in which machine
learning algorithms successfully train models to classify astronomical
images with comparable performance to humans \citep{dieleman_etal15}.

% Early efforts
Earlier efforts on automated, or ``robot'', identification of strong
lensing systems have two distinct generalized steps.  The first
enhances and extracts characteristic features, and the second uses
some form of pattern recognition in the features to classify lens and
non-lens containing systems.  Among others, selected features might
include shape parameterization
\citep[see][e.g.]{,alard_06,kuboanddellantonio_08,xu_etal16,lee_17},
colors in multi-band imaging \citep{maturi_etal14,gavazzi_etal14},
light profiles \citep{braultandgavazzi_15}, and characteristics of
potential lens galaxies \citep{marshall_etal09}.  Pattern recognition
often incorporate cutoffs in selected parameter space distributions
\citep{lenzen_etal04,joseph_etal14,paraficz_etal16}.  A number of
these have been publicly distributed, with specific end applications.
For example, {\em ArcFinder} is one such code that finds arcs in
groups or cluster scale lens \citep{seidelandbartelmann_07}, and {\em
  RingFinder} an analogous tool in searching for multiply imaged
quasars \citep{gavazzi_etal14}.  Codes like these have been
complementary to human identification \citep{more_etal16}.

% Classical machine learning methods
The latter of these methods have transitioned to using ``machine
learning'' algorithms for the pattern recognition step.  In classical
machine learning, we can train a model to separate a dataset according
to some desired classification scheme.  However, we still need to
reduce the dimensionality of the problem with a feature extraction
method.  The dominant ``features'' of that determine image
classification are not a priori known, but determined by the
algorithm.

% Recent works, neural nets, and deep learning
Much more recent works have made use of a subset tool in machine
learning, neural networks, to classify images either from input image
parameters or directly in image space.  In \citet{estrada_etal07},
authors used input shape parameters to train neural nets to identify
arc candidates.  \citet{agnello_etal15} used neural networks trained
on data from multiband magnitudes, and \citet{bom_etal16} used
extracted morphological parameters.

% Convolutional and residual
The most advanced use of neural networks for strong lensing
classification operate directly in image space.
\citep{petrillo_etal17} used mock Kilo Degree Survey data to train
convolutional neural networks, with a training set size of six million
images. \citet{lanusse_etal17} used state-of-the-art residual learning
to also work directly in the image space with minimal image
pre-processing.  A major strength of the \citet{lanusse_etal17}
implementation is that residual learning has been found to be easier
to train in deep neural networks, and can outperform implementations
of convolutional neural networks.

%  Relevance of our work
While there has been a recent surge in the use of neural networks
applied to astronomy image classification problems, \wording{these
  implementations are not readily scalable, nor are the necessary
  computational resources and expertise easily accessible to the
  entire scientific community.}  We present a supervised
classification pipeline built with open source tools, tested on mock
HST and LSST data, which will also be made publicly available.  Our
goal is to centralize an open source pipeline that is general enough
for the user to select or add appropriate image processing techniques
to augment the features of the lensed image, train and test data with
different machine learning techniques, and to select the sequence of
methods that provides optimal model performance.  Results are
reproducible on personal computers, as both the pipeline and the data
will be publicly distributed.

% Summary of paper 
This first paper serves as a case study for our pipeline with one
feature extraction and classical machine learning method,
respectively, the {\em histogram of oriented gradients} (HOG) and {\em
  logistic regression} (LR).  HOG is a fast feature extraction
procedure that quantifies edges in images, commonly used to identify
humans in security software.  LR is a linear classifier, making its
scalability relatively straightforward with existing open source
tools.  This framework is easily extendable to other feature methods
and machine learning algorithms for supervised classification.

% Summary of results
We show results of the pipeline on mock galaxy-galaxy lens systems
observed by HST and LSST as respective examples of classifier
performance on optical space- and ground-based observations.  We train
and test our pipeline on subsamples from 10,000 mock observed strong
lensing systems and 10,000 non-strong lensing systems, each centered
on a potential lens galaxy.

Our paper is organized as follows. In Section~\ref{sec:methods} we
briefly describe the methods to generate the mock HST and LSST data
and our overall image processing and classification pipeline. We
present our results in Section~\ref{sec:results}, and our summary and
discussions in Section~\ref{sec:conclusions}.

%-------------------------------------------------%
\section{Methodology}
\label{sec:methods}
%-------------------------------------------------%
\subsection{Mock Observations}
%-------------------------------------------------%
\subsection{Mock Images}
%-------------------------------------------------%
% Summary of HST and LSST images
We use mock {\em Hubble Space Telescope}\footnote{\todo{hst...org}}
(HST) and {\em Large Synoptic Sky
  Telescope}\footnote{\todo{lsst....org}} (LSST) images generated with
PICS (Pipeline for Images of Cosmological Strong lensing)
\citep{li_etal16} and LensPOP \citep{collett_15} to train and test our
model.  Along the line of sight galaxies and foreground stars.  Half
of the images used are strong lensing systems.  Lensed galaxy images
are ray-traced images of actual galaxies extracted from deep Hubble
Space Telescope observations.  The final mock images are ``observed''
with a realistic point spread function corresponding to modeled
detector artifacts for bright stars.  We refer to \cite{li_etal16} for
further details, but outline the process here.

We generate 10,000 lens containing mock observations and 10,000
non-lens containing mock observations to run our parameter search, and
keep a hold-out test set of 1,000 lens containing mock observations
and 1,000 non-lens containing mock observations on which we evaluate
the final trained model.

Lens containing mock observations include the lens galaxy, lensed
images of the source galaxy, and galaxies along the line of sight.
Non-lens containing mock observations include all but a lensed source
image.  We convolve each of the $2\times 10,000$ train/test images and
$2\times 1,000$ hold-out images to produce three separate sets of mock
``observations'' with equal numbers of lens and non-lens containing
systems: (1) HST-like observations, (2) LSST-like observations over
the span of one year, and (3) LSST-like observations over the span of
ten years.  We respectively label these observations as HST, LSST1,
and LSST10.
%-------------------------------------------------%
\subsubsection{Modeling Images of the Source Galaxies}
%-------------------------------------------------%
To produce images of the simulated lensed source galaxies, we first
set mass models of the lens galaxy to be the Singular Isothermal
Ellipsoid (SIE):
%-------------------------------------------------%
\begin{equation}\label{eqn:sie}
\kappa = \frac{\theta_{\rm E}}{2}\frac{1}{\sqrt{x_1^2/q+x_2^2 q}},
\end{equation}
%-------------------------------------------------%
where $\theta_{\rm E}$ is the Einstein radius and $q$ is the axis
\wording{ratio between semi-major and semi-minor axes $x_1$ and
  $x_2$}. We can calculate the Einstein radius from the redshift of
the lens, the redshift of the source, and the velocity dispersion of
the lens galaxy.
%-------------------------------------------------%
\begin{equation}\label{eqn:einsteinradius}
\theta_{\rm E} = 4\pi\left(\frac{\sigma_v}{c}\right)\frac{D_{ls}}{D_{s}}. 
\end{equation}
%-------------------------------------------------%
Here, $c$ is the speed of light, $\sigma_v$ is the velocity dispersion
of the lens galaxy, and $D_{ls}$ and $D_{s}$ are the respective
angular diameter distance from the source plane to the lens plane and
from the source plane to the observer.

To rotate the lenses with a random orientation angles, we adopt the
transformation below for orientation angle, $\phi$:
%-------------------------------------------------%
\begin{eqnarray}
\label{eqn:rotation}
\begin{bmatrix}

    x_{1}\prime      \\
    x_{2}\prime 
\end{bmatrix}
= 
\begin{bmatrix}
    \cos \phi  &  -\sin \phi      \\
    \sin \phi  &  ~\cos \phi      
\end{bmatrix} 
\begin{bmatrix}
    x_1      \\
    x_2      
\end{bmatrix} .
\end{eqnarray}
%-------------------------------------------------%
From Equations~\ref{eqn:sie}~-~\ref{eqn:rotation}, there are five
independent parameters that uniquely define the lensing system:
$\sigma_v$, ellipticity $q$, orientation angle $\phi$, redshift of the
lens $z_l$, and redshift of the source $z_s$.

% Parameters of lensing system
We choose $\sigma_v$, $q$, and $\phi$ randomly from typical ranges of
a galaxy, where $\sigma_v = [200, 320] km/s$, $q = [0.5, 1.0]$, and
$\phi = [0, 360]$.  The redshift of the lens galaxy is drawn from the
distribution \todo{found in \citet{???}}.  We fix the redshift of the
source to $z_s = 1.0$. We then extract images of 33 bright galaxies
from the CANDELS survey as source galaxies.  We place the source
galaxy images in projected positions close to the caustic of the
lensing system and fix the projected position of the lens galaxy to
the center of the field of view of each image.  Once we have
determined all parameters of the lensing system, we perform
ray-tracing simulations to produce the final lensed image.

%-------------------------------------------------%
\subsubsection{Modeling Lens Galaxies}
%-------------------------------------------------%
We model the light distributions of the lens galaxies with an
elliptical Sersic profile,
%-------------------------------------------------%
\begin{equation}\label{eqn:sersic}
I(R) = I_{\rm eff}~{\rm exp} \left\lbrace -b_{n} \left[ \left(
  \frac{R}{R_{\rm eff}}\right)^{1/n} - 1 \right ] \right\rbrace,
\end{equation}
%-------------------------------------------------%
where, $R = \sqrt[]{x_1^2 /q+x_2^2 q }$, $R_{\rm eff}$ is the
effective radius in units of arcseconds, $I_{\rm eff}$ is the
intensity at the effective radius, $n$ is the index of the Sersic
profile, and $q$ is the ellipticity of the lens galaxy.  We perform a
similar transformation as Equation~\ref{eqn:rotation} to orient the
sources.  We assume that the distribution of light is following that
of mass, so ellipticity and orientation are the same as the SIE model.

We use the COSMOS morphological
catalog \footnote{\url{https://arxiv.org/pdf/1501.04977v2.pdf}} to
match the velocity dispersion with an assigned effective radius,
effective luminosity, and index to the light profile.  We also assume
the light center is on top of the mass center, creating noiseless
images of lens galaxies.

We construct galaxies along the line of sight by cutting light-cones
from the \todo{Hubble Ultra Deep Field\footnote{udf...org}}.  We stack
these images with the lens galaxy image and the lensed source galaxy,
calibrating the magnitude of all three components.

%-------------------------------------------------%
\subsubsection{Producing Mock Images}
%-------------------------------------------------%

The final step is to perform a mock observations.  For HST-like
observations, galaxies along the line of sight already include noise
and a point spread function (PSF).  We therefore stack and magnitude
calibrate all components to produce the final HST-like images, which
are $300\times300~{\rm pix}^2$ with $0.03$~arcsec per pixel.

To produce LSST-like images, we use {\em LensPop} to serially perform
re-binning, convolve with the corresponding PSF, and add
noise\citep{collett_15}.  We created one year LSST-like (LSST1) images
and stacked ten-year LSST-like images (LSST10) to investigate the
performance and usability of our pipeline in future LSST
data. LSST-like images are $45\times45~{\rm pix}^2$ with $0.2$~arcsec
per pixel.

Figure~\ref{fig:mockimages} illustrates sample mock observations with
a strong lensing signature from each telescope.  The left-most column
respectively correspond to a mock HST lensing system with a highly
magnified source galaxy (top) and a less visible image of the source
galaxy (bottom). For the HST-like dataset, many arcs are visually
obvious due to the exquisite spatial resolution.

The middle column corresponds to the same simulated systems, as
observed with over ten years of LSST, LSST10.  The right-most column
corresponds to the simulated systems observed over just one year of
LSST1.  LSST10 images visually exhibit an improved signal to noise
ratio, recovering the arc feature, albeit at a much lower resolution
than with the HST-like image.  The top images of LSST10 and LSST1 show
a visible lensed source galaxy image.  The bottom images are washed
out in the bottom row, where the magnification of the source galaxy is
not as large.  The ground based noise, PSF, and limited resolution of
the LSST1 make visual giant arc identification difficult, except in
systems with the most magnified source galaxies.

%%%%%%%%%%%%%%%%%%%%%%%%%%%%%%%%%%%%%%%%%%%%%%%%                                        
\begin{figure*}[t]\label{fig:mockimages}
\begin{center}
\includegraphics[width=2\columnwidth]{figures/tiles/compilations/telescope_compilation.pdf}
\caption{Left to right show example mock HST, LSST 10 year, and LSST 1
  year images.  The top row corresponds to a lensing system with a
  very visible arc signature, and the bottom row to a lensing system
  that is less obvious.  Example mock HST images have
  $n_\text{pix}\times n_\text{pix}=300\times300$.  Example mock LSST
  images have $n_\text{pix}\times n_\text{pix}=45\times45$.  The
  resolution and noise of a ground based telescope is noticeably
  worse. Visual identification of giant arcs in the LSST images in the
  bottom row is very difficult.}
\end{center}
\end{figure*}
%%%%%%%%%%%%%%%%%%%%%%%%%%%%%%%%%%%%%%%%%%%%%%%%            

Our mock observations also include non-lens containing images.  The
procedure is similar to mock lensed images but we do not perform
ray-tracing, so these images do not have lensed source galaxies.  

Furthermore, we investigate the influence of light from the lens
galaxies on the performance of our lens identification pipeline.  We
generate another set of each HST, LSST1, and LSST10 images without the
lens galaxy.  We label these nHST, nLSST1, and nLSST10.

The final data is then comprised of $6\times10,000$ lens containing
images, and $6\times 10,000$ non-lens containing images.  We also keep
a hold-out set of $6\times 1,000$ lens containing images, and $6\times
1,000$ non-lens containing images.
%-------------------------------------------------%
\subsection{Strong Arc Lensing Identification Pipeline}
%-------------------------------------------------%

To perform our analysis, we have used tools from {\em Scikit-learn}
\citep{pedregosa_etal12}.  We outline the identification pipeline in
Figure~\ref{fig:pipeline}, which is a general description of
supervised classification.  Supervised classification is a class of
machine learning where the labels of classification in the training
set are known.  In our case, the labels are lens and non-lens
containing images.

The first step of our pipeline consists of a feature extraction stage,
where our feature vector is a {\em histogram of oriented gradients}
\citep{dalalandtriggs_05} that quantifies edges in the image.  We
describe the method and parameter search in Section~\ref{sec:hog}.  We
then use {\em Logistic Regression} (LR), a machine learning algorithm
described in Section~\ref{sec:LR}, to train a classifier model on a
subset of our images.  LR requires a parameter search over the
regression coefficient, $C_{LogReg}$, which we explore in
Section~\ref{sec:regularization}.  We briefly comment that our initial
tests with {\em Support Vector Machine} (SVM) as an alternative
machine learning algorithm yielded negligible performance improvement,
and significantly increased computation cost.  This indicated that the
features of lens and non-lens images are relatively well separated by
hyperplanes in feature space.  For these reasons, we do not include
SVM in our final analysis and comparisons and continue all discussions
with a linear classifier.

Both the feature extrator, HOG, and the linear classifier, LR, contain
parameters, which must be tested and optimized for peak model
performance.  We use {\em GridSearchCV} from {\em Scikit-learn} to
select cross-validated parameters, and discuss this step of our
methodology in Section~\ref{sec:gridsearch}.

The second step of our analysis is to test our trained model on an
independent subset of the images to assess the model performance.
Here, we evaluate the model on each test image, predicting a
likelihood (``score'') between 0 and 1 that this is a lens containing
system.  This ``holdout set'' is not used in any of our parameter
searches to keep our test metric independent of tuning.

%%%%%%%%%%%%%%%%%%%%%%%%%%%%%%%%%%%%%%%%%%%%%%%%                                        
\begin{figure}[t]\label{fig:pipeline}
\begin{center}
\includegraphics[scale=0.55]{figures/supervised_classification_workflow.png}
\caption{Cartoon of pipeline \footnote{from
    http://www.nltk.org/book/ch06.html} for supervised classification.
  In our case, the labels are either lens or non-lens containing
  images.  Our input is the set of mock HST- or LSST-like
  observations.  The feature extractor is the histogram of oriented
  gradients, producing an N-dimensional feature vector that quantify
  edges in our images.  Our machine learning algorithm is logistic
  regression.}
\end{center}
\end{figure}
%%%%%%%%%%%%%%%%%%%%%%%%%%%%%%%%%%%%%%%%%%%%%%%%            

%-------------------------------------------------%
\subsection{Feature Extractor: Histogram Oriented Gradients}\label{sec:hog}
%-------------------------------------------------%
% HOG intro - 
Originally created for human detection in computer vision, histogram
oriented gradients (HOG) is a feature extraction method that computes
centered horizontal and vertical gradients. HOG is relatively robust
to noise in the image, and is a fairly fast transform that describes
edges.  Details can be found in \citet{dalalandtriggs_05}, but we
describe the procedure here.  The end result of HOG is a one
dimensional histogram computed as follows.

% Division of the image  
HOG first divides the image into blocks of \wording{50\% overlap}.
Each block contains $m\times m$ cells-per-block that each contain
$n_{pix}\times n_{pix}$ pixels-per-cell.  The computed gradient
orientation is quantized into $N_{orient}$ bins.

% Initial binning
The orientation gradient of all pixels within each cell are binned
into the quantized orientations, providing a net gradient description
within that cell.  As an example, for $N_{orient}=3$, our bins are
centered at $\theta=0, 2\pi/3, 4\pi/3$ in radians.  If a cell has no
gradient in the vertical direction, and only a gradient across the
horizontal direction, say in the positive x-direction with
$\theta=\pi/2$, it will contribute 75\% of its magnitude to the
$\theta=2\pi/3$ bin, and 25\% of its magnitude to the $\theta=0$ bin.
The bins in all cells are then concatenated to make a larger feature
vector that is $N_{orient}$\times$N_{cells}$.

% Normalization
The last step is a normalization procedure to control for illumination
effects.  Here, the sub-histograms of each cell within the same block
are normalized with respect to one another before the transformation
returns the final feature vector.  The division of the image and the
quantization of orientations provide a total of 3 parameters in HOG:
$N_{orient}$, cells-per-block, and pixels-per-cell.  We discuss how we
select parameters using cross-validation in
Section~\ref{sec:gridsearch}

%-------------------------------------------------%
\subsubsection{Optimized Pipeline Parameters with a Grid Search}\label{sec:gridsearch}
%-------------------------------------------------%

\begin{table}
\caption{Grid Search of Pipeline Parameters}
\begin{center}
\begin{tabular}{cccccc}
\hline \\ [-0.2cm]
N$_{orient}$ & Pixels/Cell & Cells/Block & C$_{reg}$ & HOG size & Score \\ [0.2cm]
\hline \\ [-0.2cm]
\multicolumn{6}{c}{(a) HST-like data} \\ [0.2cm]
\hline \\ [-0.2cm]
9 & (4, 4) & (3, 3) & 50. & &$0.76181\pm0.00491$\\ [0.2cm]
9 & (4, 4) & (1, 1) & 50. & &$0.83403\pm0.00547$\\ [0.2cm]
9 & (8, 8) & (3, 3) & 50. & &$0.85417\pm0.00851$\\ [0.2cm] 
9 & (8, 8) & (1, 1) & 50. & &$0.86389\pm0.00687$\\ [0.2cm] 
9 & (16, 16) & (1, 1) & 50. & & $0.88542\pm0.00680$ \\ [0.2cm]
9 & (16, 16) & (3, 3) & 10. & &$0.89306\pm0.00687$ \\ [0.2cm]
8 & (16, 16) & (3, 3) & 50. & &$0.90000\pm0.00170$ \\ [0.2cm]
9 & (16, 16) & (3, 3) & 50. & &$0.90139\pm0.00804$ \\ [0.2cm]
\hline \\ [-0.2cm]
\multicolumn{6}{c}{(b) LSST-like data} \\ [0.2cm]
\hline \\ [-0.2cm]
& & & & &\\
& & & & &\\ [0.2cm]
\hline
\end{tabular}
\end{center}
\label{tab:gridsearch}
\tablecomments{Panel (a) shows the results of a grid search across a
  range of HOG parameters and regression coefficient \wording{with
    training and test sets of size 1440 with the HST-like mock
    dataset.}  Panel (b) shows the corresponding results for training
  and test sets of \wording{size 5760 with the LSST mock dataset.}  We
  explore different HOG parameters in each dataset due to resolution
  and image size differences.}
\end{table}

We run a grid search across parameters that should reasonably sample
the arc edges in either the HST- or LSST-like mock observations, and
illustrate the results in Table~\ref{tab:gridsearch}.  Recall, the
HST-like images are $300\times300$ pixels per image, while the
LSST-like mock observations are $45\times45$ pixels per image.

We first estimate the size of a cell that will contain a coherent arc
feature.  To first order approximation, subdivisions of cells that are
1/100th the area of the entire image should contain coherent arc edges
that span an elongated shape within arc-containing cells.  Therefore,
we sample the pixels per cell parameter from (8, 8) to (32, 32) for
the HST-like images, and (3, 3) to (5, 5) for the LSST-like images in
our grid search.

Next, the cells per block parameter determines the normalization of
each cell with respect to the neighboring cell.  In general, this will
downweight arc-like edges in cells that neighbor very bright cells,
such as cells that cover the central lens galaxy.  We therefore vary
the cells per block parameter between (2, 2) and (4, 4) for the
LSST-like images and between (1, 1) and (4, 4) for the HST-like
images.

The number of orientations will determine the sampling of rounded
edges.  For example, if we only have two orientations, an arc-like
feature in a cell directly north of the lensing galaxy will appear in
our HOG visualization as a strong horizontal line (e.g. see top left
in Figure~\ref{fig:hogimages}), and an arc-like feature north-east of
the lensing galaxy will appear as an L-shape.  However, contributions
from a cluster or line-of-sight galaxy in the same cell will tend to
contribute edges in all orientations of the histogram (e.g. bottom
right in Figure~\ref{fig:hogimages}).  

Finally, the resolution of the overall image will also limit the
additional information that an increase in $N_\text{orient}$ will
provide.  The best case number of orientations for each dataset is
$N_\text{orient,HST}=9$, $N_\text{orient,LSST10}=3$, \todo{and
  $N_\text{orient,LSST1}=$.}

% Impact on image performance
The image resolution affects the length of the HOG feature vector,
which has a monotonically increasing relationship with the time
required to train the model.  Additionally, for fixed memory
restrictions, there is a tradeoff between the length of the feature
vector and the size of the training set.  We will discuss how the
training set size affects the train time for each data set in
Section~\ref{sec:trainsetsize}.

%%%%%%%%%%%%%%%%%%%%%%%%%%%%%%%%%%%%%%%%%%%%%%%%                                        
\begin{figure*}[t]\label{fig:hogimages}
\begin{center}
\includegraphics[width=2\columnwidth]{figures/tiles/compilations/hog_telescope_compilation.pdf}
\caption{Left to right show example image transforms of mock images
  from Figure~\ref{fig:mockimages} with a visualized histogram of
  oriented gradients.  The image transform picks up edge features,
  with arc features showing up as edges across radial orientations.
  \wording{Each of the oriented gradients within a cell is
    color-coded by magnitude, and represented as a line in the
    direction perpendicular to that gradient.}  The actual extracted
  features fill a one-dimensional feature vector comprised of the
  magnitudes of each of the oriented edges within the visualized
  cells.}
\end{center}
\end{figure*}
%%%%%%%%%%%%%%%%%%%%%%%%%%%%%%%%%%%%%%%%%%%%%%%%            
%% \caption{Top row: Visualized HOG feature vectors of the sample
%%   lensed HST-like image from Figure~\ref{fig:mockimages}.  Left:
%%   The corresponding HOG visualization with N$_\text{orient}=5$,
%%   (32, 32) pixels per cell, and (1, 1) cells per block.  Right: The
%%   HOG visualization of the same mock image with
%%   N$_\text{orient}=9$, (16, 16) pixels per cell, and (1, 1) cells
%%   per block.  The brightest lines align with prominent edges.
%%   Note, while fewer pixels per cell will better resolve the arc
%%   feature, the ``edge'' corresponding to the arc is less prominent.
%%   Bottom row: Visualized HOG feature vectors for the LSST-like
%%   single epoch and stacked images with the best scoring HOG
%%   parameterization of N$_\text{orient}=4$, (3, 3) pixels per cell,
%%   and (3, 3) cells per block.  The single epoch image (left) does
%%   not have significant enough edge features to emerge in the HOG,
%%   but the stacked image (right) does.}  \end{center} \end{figure*}
%%   %%%%%%%%%%%%%%%%%%%%%%%%%%%%%%%%%%%%%%%%%%%%%%%%


%-------------------------------------------------%
\subsection{Machine Learning Algorithm: Logistic Regression}\label{sec:LR}
%-------------------------------------------------%
The problem of detecting gravitational lenses in images falls under
the general category of {\em{classification}} in machine learning. In
general the task is to find a function that assigns data points $x$ to
one of two or more classes, denoted by the class label $y$.  This is
equivalent to specifying a decision boundary, or decision boundaries
between the classes in the space of the data points.  (Compare this to
{\em{regression}} in which the task is to find a function $y=f(x)$,
where $y$ is a continuous, rather than discrete, variable.) In our
case, we have two classes: lens and non-lens containing images, and
the data points $x$ are the HOG feature vectors transformed from the
training set data.  In this paper we use the Logisitic Regression (LR)
algorithm, for which the decision boundary is a hyperplane .  (The
equivalent in the regression setting would be linear regression.) In
LR we determine the optimal hyperplane by minimizing the objective
function
\begin{equation}
\label{eqn:log_reg_objective}
L(A,b) = \sum_i \log \left[1 + \exp \left(-y_i (A \cdot x_i + b) \right) \right]
\end{equation}
where $x_i$ is a data point (HOG feature vector), $y_i$ is the
known label for that data point ($1$ for a lens containing
image, $-1$ for a non-lens containing), and $A$ and $b$ are
the parameters of the hyperplane. Eq. \ref{eqn:log_reg_objective}
is to be minimized with respect to $A$ and $b$.
Other more complicated machine learning algorithms exist which
do not necessarily produce a linear decision boundary, such as
Support Vector Machines (SVM), Random Forests, and Neural Networks.
training set data \citep{hastie_09}.

The HOG feature vectors in this paper can be very high-dimensional.
When dealing with high-dimensional data, where the number of
dimensions becomes comparable to the number of data points,
overfitting can become an issue. (An example of an extreme case
of overfitting would be if you fit a degree $n$ polynomial to $n$
points.  The polynomial would simply wiggle so that it goes
through every point, and would have no predictive power if you
tried to interpolate or extrapolate.) In machine learning,
overfitting is avoided using different {\em{regularization}}
techniques. A common choice for logistic regression is to
add a penalty term to Equation~\ref{eqn:log_reg_objective}
\begin{equation}\label{eqn:log_reg_regularized}
L_\text{Reg}(A,b,C_\text{LogReg}) = L(A,b) + \frac{1}{2C_\text{LogReg}} \|A\|
\end{equation}
where the norm $\|A\|$ is typically taken as either the
sum of squares of the coefficients ($L_2$ norm) or the sum
of absolute values of the coefficients ($L_1$ norm).  In this
paper we use the former.

The amount of regularization is controlled by the parameter
$C_\text{LogReg}$: larger values of $C_\text{LogReg}$ correspond to
increasing model complexity.  If $C_\text{LogReg}$ is too large then
the model will overfit, and if it is too small the model will
underfit.  To determine whether a model is overfit or underfit, the
model is trained, (i.e. Eq. \ref{eqn:log_reg_regularized} is
minimized), on a subset of the data called the {\em{training set}} and
its performance (goodness of fit) is evaluated both on the the
training set and a set of hold-out data called the {\em{test set}}
that was not used in constructing the model.
Figure~\ref{fig:regularization} shows the performance of a model for
selected HOG parameters as a function of the regularization parameter,
$C_\text{LogReg}$, for both the training and the test set.  (The
performance can be measured by the accuracy, i.e. percent of images
correctly classified, or by some other metric, such as the area under
the Receiver Operating Characteristic curve (see
Section~\ref{sec:ROC}).)

When $C_\text{LogReg}$ is small, the performance of the model improves
with increasing $C_\text{LogReg}$ on both the training and test set,
meaning that $C_\text{LogReg}$ is
still in the underfitting regime. Eventually the performance on the
test set reaches a maximum and starts to decrease, even while the
performance on the training set continues to increase.  This means
that the model is no longer generalizing well and is starting to
overfit. The optimal $C_\text{LogReg}$ occurs when the performance
of the test set is at its maximum; this is the value of $C_\text{LogReg}$
that should be used in the final model.

In practice, there is something of a trade-off between accuracy and
computational resources because a larger value of $C_\text{LogReg}$
will also increase the training time, since a larger $C_\text{LogReg}$
corresponds to a less constrained parameter space being searched
\todo{(citation needed)}.
We discuss the performance and training time dependence on $C_\text{LogReg}$
in Section\ref{sec:regularization}.

%-------------------------------------------------%
\section{Results}
\label{sec:results}
%-------------------------------------------------%

%-------------------------------------------------%
\subsection{Receiver Operating Characteristic}\label{sec:ROC}
%-------------------------------------------------%

In this section, we discuss the Receiver Operating Characteristic
(ROC) curve (see Figure~\ref{fig:ROCcompilation}), which shows the
true positive rate (tpr) as a function of false positive rate (fpr)
for a given model and test set.  The true positive rate is defined as
the number of true positives divided by the total number of positives
(completeness of the positive identifications).  The false positive
rate is defined as the number of false positives divided by the total
number of negatives (impurity of the identifications).  The ROC curve
illustrates the performance of our trained model as we vary the
discrimination threshold.

The classifier model assigns a score to each test image, which is a
probability that the image is a strong lensing system.  To construct
the ROC curve, we rank the test images by probability, and calculate
the tpr and fpr for decreasing discrimination threshold.

For very high discrimination threshold, we have maximal purity, but a
low completeness (bottom left region of the ROC).  For a very low
discrimination threshold, we have maximal completeness, but minimal
purity (top right region of the ROC).  The ideal model would have an
ROC curve with data points that go from $(x, y) = (0, 0)$ to $(x, y) =
(0, 1)$ to $(x, y) = (1, 1)$.

In the context of strong lensing systems, we wish to maximize the true
positive rate so we have a representative count of the fraction of
strong lensing systems in an observed volume of the universe.  We also
want to minimize the false positive rate.  Positively identified
strong lensing systems will require expensive spectroscopic follow-up
for validation.  The steepness of the ROC curve indicates how well the
model will optimize the two, and we can characterize the steepness by
the area under the curve (AUC).  The ideal model would have an AUC of
1.  We show the ROC curves of our best performing models in each
dataset.

% Precision-recall
For comparison, we also show a Precision-Recall curve (see
Figure~\ref{fig:PRcompilation}), where precision is the number of true
positives identified as positive divided by the total number of
positively identified images (purity) and recall is the number of true
positives identified as positive divided by the total number of true
positives (completeness).  Each point in the figure is calculated with
a varying threshold for identification.  Since our sample has a class
balance of 50-50 between positives (lens) and negatives (non-lens),
the most lenient threshold for identification would yield a precision
of 0.5 at a recall of 1.0.  It is important to note that this figure
changes as the class balance changes; if we had 90\% non-lenses and
10\% lenses, the most lenient threshold would yield a precision of 1/9
at a recall of 1.  The ROC curve is a more typical metric for
supervised classification.

% Summary of ROC curves
Figure~\ref{fig:ROCcompilation} shows the ROC curves for models
trained using the entire 10,000 training sample, with best-case HOG
and regularization parameters.  The models have been evaluated on a
hold-out test set of 1,000 images that were not used in the parameter
search.  We show the mock HST, LSST1, and LSST10 results respectively
in red, blue, and green.  Solid lines correspond to a model trained
and tested on images with the lensing galaxy.  Dashed lines correspond
to a model trained and tested on images where the lensing galaxy is
excluded from the mock observation, simulating an ideal modeling and
subtraction of the lensing galaxy, which has been one proposed method
to improve the identification of strong lensing systems.  The
corresponding AUC is listed in the legend.

% Model performance summary
The model performance for the mock HST data is AUC=0.975 for images
with the lens galaxy, and AUC=0.98 for images without the lens galaxy
(red solid and dashed).  On the other hand, the model for our
LSST-like dataset for one year has an AUC=0.625 with the lens galaxy
and AUC=0.579 without the lens galaxy (blue solid and dashed), and the
model for our LSST-like dataset for 10 years has an AUC=0.809 with the
lens galaxy and AUC=0.792 without the lens galaxy (green solid and
dashed).  Removal of the lens galaxy does not systematically perform
better, and is actually dependent on the size of the training set.  We
discuss relative model performance and complexity for images with and
without the lens galaxy in Section~\ref{sec:trainsetsize}.  

%%%%%%%%%%%%%%%%%%%%%%%%%%%%%%%%%%%%%%%%%%%%%%%%                                        
\begin{figure}[t]
\begin{center}
\includegraphics[width=\columnwidth]{figures/metrics/ROC/ROC_compilation.pdf}
\caption{Red, blue, green: ROC curves for models trained on our whole
  10,000 training set and tested on our holdout set of 1,000.  These
  respectively correspond to the HST, LSST 1 year, and LSST 10 year
  data.  The solid lines are for data that include the lensing central
  galaxy, and the dashed lines for the data where there is no lensing
  galaxy, mimicking an ideal removal of the lens.  Model performance
  can be summarized by the area under the curve (AUC), labeled in the
  legend. $AUC=1$ is a perfect model, and $AUC=0.5$ is a useless
  model.}\label{fig:ROCcompilation}
\end{center}
\end{figure}
%%%%%%%%%%%%%%%%%%%%%%%%%%%%%%%%%%%%%%%%%%%%%%%%            
%%%%%%%%%%%%%%%%%%%%%%%%%%%%%%%%%%%%%%%%%%%%%%%%                                        
\begin{figure}[t]
\begin{center}
\includegraphics[width=\columnwidth]{figures/metrics/PR/PR_compilation.pdf}
\caption{Red, blue, green: Precision-recall (PR) curves for models
  trained on our whole 10,000 training set and tested on our holdout
  set of 1,000.  These respectively correspond to the same models and
  data shown in Figure~\ref{fig:ROCcompilation}.  An ideal model would
  reach both a precision (purity) and recall (completeness) that equal
  1.  Note, that this performance describes a data set with a 50-50
  split between lens and non-lens containing
  images.}\label{fig:PRcompilation}
\end{center}
\end{figure}
%%%%%%%%%%%%%%%%%%%%%%%%%%%%%%%%%%%%%%%%%%%%%%%%            

%-------------------------------------------------%
\subsection{Effects of Regularization on Model Performance}\label{sec:regularization}
%-------------------------------------------------% 

% Logistic regression and regularization
As described in Section~\ref{sec:LR}, LR trains a model with
complexity determined by the regularization parameter coefficient,
$C_\text{LogReg}$.  Larger values of $C_\text{LogReg}$ are less
regularized and allow for increased model complexity.  The highest
values of $C_\text{LogReg}$ will better describe features in the
training set.  However, an overly complex model will overfit the
training set at the expense of its performance on any independent test
set.  The regularization parameter ultimately defines the model
performance, and we must perform a parameter search to identify the
optimal value for $C_\text{LogReg}$.

%  Our running over C - score and AUC as characterizations
Figure~\ref{fig:regularization} shows the model performance as a
function of regularization parameter for each data set HST, LSST, and
LSST10.  The solid and dotted blue lines respectively correspond to
the model performance on the test and training set, with the AUC as a
metric for performance.  In red, we show the train time as a function
of $C_{LogReg}$.

% Training behavior with C
As expected, the training set increases and asymptotes with
$C_\text{LogReg}$.  With increasing model complexity, the model better
fits the training data set.  This is analogous to fitting a seventh
order polynomial to seven data points, where the fitting function will
go through every point but will not likely predict additional points.
With increasing model complexity, we are able to better able to
capture features that are generally characteristic of strong lensing
systems with arcs. However, past a certain $C_\text{LogReg}$, the
model performance on the test set decreases or asymptotes, as it has
overfit the training set.  We use the scaling of AUC with
$C_\text{LogReg}$ when training 8,000 out of our full 10,000 training
data set to determine the best value for $C_\text{LogReg}$.  However,
the optimal parameter is also dependent on the size of the training
set (see Section~\ref{sec:trainsetsize}), so this choice is not
generalizable.

% Train time with Creg
For fixed train size, the log of the train time roughly scales
linearly with the log of $C_\text{LogReg}$.  Since lower values of
$C_\text{LogReg}$ correspond to a more regularized model, there is a
smaller volume in hyperparameter space to search for the best fit
coefficients.  The solution, on average, will converge more quickly,
for more regularized models.  The scaling is not purely monotonic
because the fitting still has some randomness associated with the path
it takes to convergence.

% Scores vs. regularization on larger train/test set with comparable
% grid search in parameterization
%%%%%%%%%%%%%%%%%%%%%%%%%%%%%%%%%%%%%%%%%%%%%%%%                                        
\begin{figure*}[t]
\begin{center}
\includegraphics[width=.66\columnwidth]{figures/Creg/HST.pdf}
\includegraphics[width=.66\columnwidth]{figures/Creg/LSST1.pdf}
\includegraphics[width=.66\columnwidth]{figures/Creg/LSST10.pdf}
\caption{AUC of the model with varying LR regularization coefficient
  parameter, $C_{reg}$, used when training the model classifier.  We
  use a subset of the the 10,000 training images to search over the LR
  $C_{reg}$ parameter, training on 8,000 and testing on 1,000.  Each
  panel corresponds to a different mock observation.  From left to
  right: HST, LSST for one year, and LSST for 10 years.  The solid
  blue lines correspond to the AUC of the test set, and the dotted
  blue lines to the AUC of the training set.  To avoid overfitting, we
  choose the smallest parameter for which the AUC of the test set is
  maximal: 5000, 10, and 5000, respectively.  In thin red solid lines,
  we show the train time of the model, which roughly increases in a
  log-log scaling with logistic regression coefficient
  parameter.}\label{fig:regularization}
\end{center}
\end{figure*}
%%%%%%%%%%%%%%%%%%%%%%%%%%%%%%%%%%%%%%%%%%%%%%%%            

%-------------------------------------------------%
\subsection{Dataset Size Dependence}\label{sec:datasize}
%-------------------------------------------------% 
%-------------------------------------------------% 
\subsubsection{Effects of Training Set Size on Performance and Train Time}\label{sec:trainsetsize}
%-------------------------------------------------% 

In this section, we show the effects of training set size on model
performance on the holdout test set of 1,000 images.  We also compare
the improvement between images that include the lens galaxy and images
with no lens galaxy.

Figure~\ref{fig:trainsizeLSST10} shows how the AUC and the optimal
logistic regression coefficient depend on the log of the size of the
training set for both the LSST10 data in solid blue lines, and the
nLSST10 data in dashed blue lines.  The AUC for models trained on the
LSST10 data improves almost linearly with the log of the training set
size, increasing from AUC=0.705 to AUC=0.788 when the train size is
increased from 500 lens/non-lens images to 8000 lens/non-lens images.
However, for the nLSST10 data, where the lens galaxy has been removed
from the images, the improvement is less dramatic.  With the same
increase in training size, the AUC for nLSST10 changes from just below
0.77 to just below 0.78.

In solid red, we see that the optimal value for $C_\text{LogReg}$ for
LSST10 roughly scales logarithmically with the log of the train size,
with an exception of the data point corresponding to train size of
2000.  Generally, a larger training set will require an increase in
model complexity to capture additional information.  This is also true
for the optimal $C_\text{LogReg}$ dependence on the number of training
images in the nLSST10 data.  But, the required complexity is
systematically less than for the LSST10 images.  

% Comparison of the two LSST10 vs. nLSST10
In the LSST10 case, the trained model can incorporate the additional
information of the edges from the lens galaxy, which is correlated
with the lensing cross-section and likelihood of the image being a
lensing system.  The nLSST10 images do not contain this information,
but provide cleaner signals of the lensed image for lensed images that
occur close to the lens galaxy.  The cleaner signal in nLSST10 allows
for better model performance for smaller training data set sizes
($N_\text{train}\sim5,000$).  However, models trained on LSST10
improve more rapidly with train size, since the additional information
from the lens galaxy better describes the lens containing images.  For
$N_\text{train}\gsim 5,000$, the models trained on LSST10 outperform
  those trained on nLSST10.

%M Model complexity, train time, and performance
Since the models that contain information from the lens galaxy edges
in LSST10 are more complex, the models require a larger training set
size for a better fit.  While model performance for LSST10 appears to
steadily increase, this comes at the cost of increased train time,
which is two-fold.  The train time will increase due to both an
increase in data to fit, and also an increase in optimal
$C_\text{LogReg}$ where the volume of hyperparameter space for allowed
solutions is larger (see red lines in
Figure~\ref{fig:regularization}).

% Train time
Figure~\ref{fig:traintimeLSST10} shows the train time for LSST10 and
nLSST10 as a function of the size of the training set for the optimal
regularization parameter for that subset training data.  Each model
uses features extracted with the same HOG parameterization from the
grid search and the optimal regularization parameter for that subset
train data.  The train time of a given model generally increases for
increasing regularization parameter.  For the subset of train size
$N_\text{train}=2000$ in LSST10, the optimal regularization parameter
happened to be $C_{LogReg}=100$, whereas it was $C_{LogReg}=500$ for
the subset of trainsize 1000, and $C_{LogReg}=1000$ for the subset of
trainsize 4000.  This makes the train time increase at train size
$N_\text{train}=2000$ for LSST10 less dramatic than the average
log-log slope of approximately 2.

% Train time behavior for nLSST10
Since nLSST10 does not contain the lens galaxies, fewer of the
extracted features describe the lens system, requiring decreased model
complexity.  The increase in train time for nLSST10 as a function of
train size is mostly due to only having more data to fit in the
regression, leading to a steady and slow increase of train time with
number of training images with log-log slope of approximately 1.

% AUC vs. regularization on larger train/test set with comparable
% grid search in parameterization
%%%%%%%%%%%%%%%%%%%%%%%%%%%%%%%%%%%%%%%%%%%%%%%%                                        
\begin{figure}[t]
\begin{center}
\includegraphics[width=\columnwidth]{figures/trainsize/LSST10_with_nLSST10.pdf}
\caption{Solid (dotted) blue line: AUC for models with varying size of
  the training set for LSST10 (nLSST10).  Solid (dotted) red line:
  Optimal regularization parameter for training on the train size for
  LSST10 (nLSST10). The improvement of AUC scales linearly with the
  log of the training set size, but train time roughly scales
  logarithmically.  Note, LSST10 requires more model complexity to
  exceed the performance of nLSST10.}\label{fig:trainsizeLSST10}
\end{center}
\end{figure}
%%%%%%%%%%%%%%%%%%%%%%%%%%%%%%%%%%%%%%%%%%%%%%%%            

%%%%%%%%%%%%%%%%%%%%%%%%%%%%%%%%%%%%%%%%%%%%%%%%                                        
\begin{figure}[t]
\begin{center}
\includegraphics[width=\columnwidth]{figures/traintime/LSST10.pdf}
\caption{Solid (dotted) blue line: Train time for models with varying
  size of the training set for LSST10 (nLSST10).  Train time roughly
  scales logarithmically.  Note, LSST10 requires more model complexity
  to exceed the performance of nLSST10, and therefore requires more
  training time for continual increase in
  performance.}\label{fig:traintimeLSST10}
\end{center}
\end{figure}
%%%%%%%%%%%%%%%%%%%%%%%%%%%%%%%%%%%%%%%%%%%%%%%%            

%-------------------------------------------------% 
\subsubsection{Effects of Rotation on AUC}
%-------------------------------------------------% 

% Augmenting dataset
To augment our training set, we rotated each image in the set by
multiples of $90^o$.  Since our feature extraction method of HOG is
not rotationally invariant, augmenting our data by a factor of four
naturally optimizes the use of available training data.  This has an
equivalent improvement to the study illustrated in
Figure~\ref{fig:trainsizeLSST10}.

% Rotations of the test set
We also tested the effects of evaluating our model on all four
rotations of the test set, and using the average score of each test
image to calculate the AUC.  In Figure~\ref{fig:rotation_test}, we
show the AUC for different orientations of the datasets.  The x-axis
corresponds to each of our three datasets, with and without the lens
galaxy.  The y-axis shows the AUC.  The filled blue circles correspond
to the four AUCs calculated when the model evaluated images at each of
the four rotations.  The filled red stars correspond to the AUC
calculated from the average of all four test scores, which are
systematically higher than any one rotation.  The average score across
all rotations for each image is likely to be less noisy for the whole
test sample, giving an improved AUC.

% Summary figure
Figure~\ref{fig:rotation_test} also summarizes the best-case results
of our models trained on our entire 10,000 training sets, and tested
on our holdout 1,000 test set.  Recall, however, that we expect model
performance on images containing the lens galaxy to improve further
with larger training sets (see Figure~\ref{fig:trainsizeLSST10}).

%%%%%%%%%%%%%%%%%%%%%%%%%%%%%%%%%%%%%%%%%%%%%%%%                                        
\begin{figure}[t]
\begin{center}
\includegraphics[width=\columnwidth]{figures/rotated_aucs/rotated_aucs.pdf}
\caption{Our summary figure: The AUCs of models trained on the full
  10,000 and tested on the holdout 1,000.  Blue circles: AUC
  calculated from scores of images at a given rotation (e.g. 0, 90,
  180, and 270 degrees).  Red stars: AUC calculated from the average
  score of all rotations of each image.  The average score produces an
  improved AUC in all data sets.  We expect the AUC to further improve
  with increased train size.}\label{fig:rotation_test}
\end{center}
\end{figure}
%%%%%%%%%%%%%%%%%%%%%%%%%%%%%%%%%%%%%%%%%%%%%%%%            



%-------------------------------------------------%
\subsection{Image Classification Performance}\label{sec:performance}
%-------------------------------------------------%

%%%%%%%%%%%%%%%%%%%%%%%%%%%%%%%%%%%%%%%%%%%%%%%%                                        
\begin{figure*}[t]
\begin{center}
\includegraphics[width=.8\columnwidth]{figures/tiles/compilations/LSST10/tp.pdf}\hspace{5pt}
\includegraphics[width=.8\columnwidth]{figures/tiles/compilations/LSST10/fp.pdf}\\\vspace{5pt}
\includegraphics[width=.8\columnwidth]{figures/tiles/compilations/LSST10/bp.pdf}\hspace{5pt}
\includegraphics[width=.8\columnwidth]{figures/tiles/compilations/LSST10/bn.pdf}\\\vspace{5pt}
\includegraphics[width=.8\columnwidth]{figures/tiles/compilations/LSST10/fn.pdf}\hspace{5pt}
\includegraphics[width=.8\columnwidth]{figures/tiles/compilations/LSST10/tn.pdf}\\
\caption{LSST 10 year mock images. Left two columns: Lens containing
  images, annotated with the image score assigned by our trained
  classifier.  Right two columns: Non-lens containing images,
  annotated with the image score.  The top two rows show
  characteristic images that will be accepted with a high threshold
  for classification, contributing to the bottom left of the ROC curve
  in Figure~\ref{fig:ROCcompilation}.  The middle two rows show
  characteristic images that will be accepted with a moderate
  threshold, contributing to the knee of the ROC curve, with true
  positive rates and false positive rates of $tpr\approx0.8$ and
  $fpr\approx0.25$, respectively.  The bottom two rows show
  characteristic images that will only be accepted with an extremely
  lenient threshold, contributing to the top right area of the ROC
  curve.}\label{fig:ROCsamplesLSST10}
\end{center}
\end{figure*}
%%%%%%%%%%%%%%%%%%%%%%%%%%%%%%%%%%%%%%%%%%%%%%%%            

\subsubsection{Populating the ROC Curve}
% Defining the 6 paradigms of classification
In this section, we discuss the different image types that our model
is most and least able to successfully classify.  We have six
paradigms of model performance on the mock images based on the score
an image receives when evaluated by the trained model, and its true
label.  From highest scoring to lowest scoring: True Positives (tp),
False Positives (fp), Borderline Positives (bp), Borderline Negatives
(bn), False Negatives (fn), and True Negatives (tn).

Figure~\ref{fig:ROCsamplesLSST10} illustrates four images from each
paradigm for the holdout test set of LSST10.  The trained model used
the entire 10,000 training set.  The left two columns show lens
systems, and the right two columns show non-lens systems.  The
annotation in the top left corner of each images shows the score,
\todo{and the letters in the bottom right label the paradigm of
  performance.}

% tp, tn description
In general, the true positives (the highest scoring lens systems) have
lensed images with large magnification.  The the true negatives (the
lowest scoring non-lens systems) have small lens galaxies with
galaxies along the line of sight that are rounded.  The successful
classification of these two paradigms are least sensitive to the
threshold.  On the other hand the failed classification of the false
negatives (the highest scoring non-lens systems) and the false
positives (the lowest scoring lens systems) are also least sensitive
to the threshold.  False negatives are typically lens systems with
lensed images of smaller magnification and minor distortions that
mimic along the line of sight galaxies that the model has learned to
ignore.  False positives are often non-lens systems with elongated,
elliptical, or ``fuzzy'', galaxies along the line of sight whose signal
blends with the lens galaxy contributing to the false positive rate
even for conservative thresholds.  Visually, these false positives are
virtually indistinguishable from true arcs, and would require
spectroscopic follow-up.

% Borderlines
The middle two rows of Figure~\ref{fig:ROCsamplesLSST10} illustrate
the borderline positives and borderline negatives. The successful
classification of the borderline positives and negatives are most
sensitive to the threshold, and would be the first candidates for
alternative classification methods, such as visual follow-up.
Thresholds set around these scores yield a true positive rate and
false positive rate of $tpr\approx0.8$ and $fpr\approx0.25$,
respectively.

%%%%%%%%%%%%%%%%%%%%%%%%%%%%%%%%%%%%%%%%%%%%%%%%            
\subsubsection{Dependence on Lens-Model Parameters}
%%%%%%%%%%%%%%%%%%%%%%%%%%%%%%%%%%%%%%%%%%%%%%%%                                        

% Summary of subsection
Here, we examine lens-model parameters that affect how well our
pipeline can classify the system.  The lens-model parameters we
examined are the redshift, ellipticity, orientation angle, and
velocity dispersion of the lensing galaxy, and also the magnification
of the lensed image compared with its original size.  We found that
the magnification of the lensed image is the most correlated
lens-model parameter with our trained model performance, with a
secondary and related correlation with lens galaxy velocity
dispersion.  The more strongly lensed an image is, the larger its
magnification, and the easier it is for our trained model to classify.

% Discuss figure
In Figure~\ref{fig:scorevslensparams}, we show the classification
score as a function of image magnification for lens images in our
HST-like sample.  The left-hand side shows the relation color-coded by
the redshift of the lensing galaxy, and the right-hand side shows the
relation color-coded by the velocity dispersion of the lensing galaxy.
Lensing systems with images that have magnification $\gsim7$ will
almost certainly be classified as positive with threshold scores above
$\gsim0.5$.  These systems are also those that are most easily
classified by eye.  However, our trained model has varying performance
for systems with lower magnifications.



% Some dependence on redshift and velocity dispersion, which is folded
% into magnitude
Almost \todo{X\%} of lensing systems with magnifications $\lsim7$ will
be falsely identified as non-lenses for classification threshold
scores $\gsim0.5$.  These systems typically have lenses that are at
higher redshifts of $z\gsim0.5$ (lighter colors in left figure) and/or
velocity dispersions lower than $\sigma_v\lsim230.$~km/s (darker
colors in right figure).  Note that lensing galaxies with smaller
velocity dispersions are less massive and therefore have a smaller
efficiency of lensing cross section.  The smaller efficiency means
that a background galaxy is less likely to be strongly lensed with
high magnification.  Also, due to hierarchical structure formation,
galaxies are less massive at higher redshifts, so the trends of model
performance with these three parameters are somewhat degenerate with
one another.  

The relationship between our model performance and these lens
parameters indicates that the magnification of lensed images is the
most relevant, and the distribution of image magnification in a data
set will impact trained model performance.  We did not find
correlations in model performance with other lens parameters.  It is
also useful to keep in mind that lensed galaxy images with
magnification $\lsim5$ (e.g. slightly stretched spherical galaxies)
are often visually indistinguishable from elliptical galaxies along
the line of sight.  

In a forthcoming paper, we will discuss how class imbalance, or
differences between the lens-model parameter distributions in the
training and test sets affect model performance and will explore a
method to correct for this.
%%%%%%%%%%%%%%%%%%%%%%%%%%%%%%%%%%%%%%%%%%%%%%%%                                        
\begin{figure*}[t]
  \centering 
  \mbox{
    \subfigure{\includegraphics[width=\columnwidth,trim=0 0 0 0,
        clip]{figures/HST/or5ppc32cpb4/magnitude_d_colorcodedby_z.pdf}}\hfill
    \subfigure{\includegraphics[width=\columnwidth,trim=0 0 0 0,
        clip]{figures/HST/or5ppc32cpb4/magnitude_d_colorcodedby_velocity_dispersion.pdf}}
  } 
\caption{Image classification score on HST mock data as a function of
  lensed image magnification.  Left color-coded by redshift, right is
  color coded by the lens velocity dispersion.}
\label{fig:scorevslensparams}
\end{figure*}
%%%%%%%%%%%%%%%%%%%%%%%%%%%%%%%%%%%%%%%%%%%%%%%%                                        

%-------------------------------------------------%
\section{Summary and Discussions}
\label{sec:conclusions}
%-------------------------------------------------%

% Summarize paper
We have presented a supervised classification pipeline to
automatically identify galaxy-galaxy strong lensing systems using a
histogram of oriented gradients (HOG) as a feature extractor, and
logistic regression (LR) as a machine learning algorithm.  Our
pipeline can easily be extended to identify other strong lensing
features, such as multiply imaged quasars, and to test alternative
features and/or machine learning algorithms.


% Training set
We have also made use of a new sophisticated set of mock observations,
which will also be made publicly available.  The lensing systems have
lens galaxies generated with a realistic redshift distribution and
along the line of sight galaxies drawn from Hubble Ultra-Deep field
observations.  We have explored results from mock HST, one year LSST,
and ten year LSST observations.


We summarize key points below:

%%%%%%%%%%%%%%%%%%%%%%%%%%
\begin{itemize}

\item We have designed our pipeline to easily select and add image
  pre-processing and feature extraction methods, and to select a
  machine learning algorithm for classification.  Additionally, the
  user can easily perform parameter searches to train a model with the
  best parameters for a given problem.

\item We have tested and run parameter tests for a histogram of
  oriented gradients as an efficient and effective feature extractor
  for galaxy-galaxy strong lensing systems in both a space based
  (HST-like) and ground based (LSST-like) observation.  We have also
  tested and run parameter tests for logistic regression as a
  scalable, cheap, and effective machine learning algorithm.

\item We find AUC values of ROC curves of optimized classifier models
  to yield $AUC=0.975$ for the HST-like data, $AUC=0.809$ for the
  stacked LSST-like data.  Model performance exhibits continual
  increase with the training size.

\item While removal of the lens galaxy improves model performance for
  smaller size training samples, features from the lens galaxy improve
  model performance for larger training data sets.

\item Images that were easiest for our model to classify typically
  were lens systems that had high lensed image magnification and a
  lens galaxy with large velocity dispersion or non-lens systems with
  lens galaxies with smaller velocity dispersion and non-elongated
  along the line of sight galaxies.
\end{itemize}
%%%%%%%%%%%%%%%%%%%%%%%%%%

% Benefits of our method
We emphasize that classical linear classifiers, such as logistic
regression, is scalable and relatively easy to parallelize with open
source tools such as \todo{\em Spark}\footnote{...spark...org}}.
  Initial checks indicated that HOG features are well separated enough
  for a linear classifier algorithm.  Reproducible results do not
  require supercomputing resources.

% Impact of work
These results indicate that effective image classification of
galaxy-galaxy strong lensing systems do not require large scale
computing resources.

% Caveat of the mock data set
One major caveat of our resuls is the fact that our mock data likely
does not describe the full range of lens and non-lens images that will
be observed.  For example, the along the line of sight galaxies in our
training and test images all come from the Hubble Ultra Deep field,
sampling a smaller range of potential systems that are not associated
with a lensed image.  We will address these shortcomings in subsequent
work.  

% Interpreting TP and FP rates
Finally, we must take the class balance, or relative number of lens to
non-lens systems, into account when assessing a metric.  The
Purity-Recall (purity-completeness) metric is sensitive to the ratio
of lens and non-lens systems in the data.  Both of our training and
test data sets have $50\%$ lensed and $50\%$ unlensed images, which is
not expected in observations.  The ROC curve metric is insensitive to
the ratio, but is sensitive to the sampling.  Given alternative lens
and non-lens sample splittings, our true positive and false positive
rates in the ROC curves would stay the same, making the ROC curve a
more standard metric in the literature.  On the other hand, the ROC
curves show a representative rate for lens and source distributions
that are evenly sampled.  We do {\it not} expect this sampling to be
representative of what we might expect from a an observational survey.


\acknowledgments CA acknowledges support from the Enrico Fermi
Institute at the University of Chicago, and the University of Chicago
Provost's Office. This work was also supported in part by the Kavli
Institute for Cosmological Physics at the University of Chicago
through grant NSF PHY-1125897 and an endowment from the Kavli
Foundation and its founder Fred Kavli.
\lastpagefootnotes


%%%%%%%%%%%%%%%%%%%%%%%%%%%%%%%%%%%%%%%%%%%%%%%%                                        
%%%%%%%%%%%%%%%%%%%%%%%%%%%%%%%%%%%%%%%%%%%%%%%%                                        
%\appendix
%%%%%%%%%%%%%%%%%%%%%%%%%%%%%%%%%%%%%%%%%%%%%%%%                                        
%\section{Sample stacked mock LSST images}
%%%%%%%%%%%%%%%%%%%%%%%%%%%%%%%%%%%%%%%%%%%%%%%%                                        
%\subsection{BCG included}
%%%%%%%%%%%%%%%%%%%%%%%%%%%%%%%%%%%%%%%%%%%%%%%%                                        

\bibliography{ms}

\end{document}  

