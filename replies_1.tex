\documentclass{article}
%\usepackage{pdftex}
%\usepackage{epsf}
\usepackage{graphicx}
%\input epsf

% less/greater or approximately equal definitions
%\def\lesssim{\mathrel{\mathpalette\fun <}}
%\def\gtrsim{\mathrel{\mathpalette\fun >}}
%\def\lesssim{ _<\atop{^\sim}}
%\def\gtrsim{ _>\atop{^\sim}}
%\def\Msun{M$_{\odot}${\ }}
%\def\Mvir{M$_{vir}${\ }}
%\def\rvir{$r_{vir}${\ }}

\oddsidemargin 30pt
\evensidemargin 30pt
\marginparwidth 29pt
\textwidth 427pt
\headsep 0pt
\headheight 0pt
\textheight 680pt
\topmargin 0pt
\parskip 5pt
\parindent 0pt

\begin{document}

\vspace{0.5cm}

{\bf 1) Simulated sources, 2.1.3. Why are your sources all at redshift 0.2?!? that's
in front of all your lenses - so what's going on? Is that a typo for zs=2
(which would be a fair enough starting point) Otherwise sources at zs=0.2
will be much too large.}

Yes, thank you for pointing this out - it is a typo for zs=2.0.
Highlighted changes in bold.


{\bf 
2) Do you k-correct the magnitude and colours of the sources when you
change their redshifts?
}

Nan had corrected the size a magnitude of the sources corresponding to
the change in cosmological distance, Ds, as their redshifts were
adjusted to zs.  However, all simulated images are monochromatic, and
colors were not shifted.  While the distribution of lens properties
are already not realistic (e.g. constant source redshift), they
provide a sufficient start in covering feature space (los+lens+source
image galaxies) to test supervised machine learning methods and
particularly methods that use edge features.  We have altered the
first paragraph of 2.1.4 for clarity, and changes are in bold.


{\bf 
a) 'positions near the caustics' is too vague - what range of source positions
do you allow? are you focussing solely on lenses with fairly high
magnification?
}

Thank you for pointing this out.  The sources are located within a
square box centering at the center of the images (mass center as
well). The size of the box is $(-1'', 1'')\times(-1'', 1'')$ (a
$2''\times2''$ box).  This information is added in bold in 2.1.3.

We use the same auxiliary response file (ARF) and redistribution matrix file
(RMF) for both generating photon maps and in generating best fit
spectra, and have included this information in our description of both
photon map generation and spectral fitting.

{\bf 4) You should point out that the CADNELS sources are observed
  under the HST seeing, but lensing magnifies small regions. You
  therefore can't simulate the clumpiness that HST imaging of lensed
  arcs tends to show.  }

You are right on the seeing of CANDELS sources, clumpiness will not be
presented in the lensed images. and this is one shortage in the
simulation.  This has been added in bold in the discussion of caveats
in section 4.

{\bf 5) For the same reason, you do need to convolve the simulated HST
  lensed images with a HST psf, and you need to sort out the noise
  properties. The first paragraph of 2.1.4 has me very confused as to
  what you're actually doing, and if it's right or not.}

To clarify the first paragraph of 2.1.4 first, the mock images combine
four components from the field of view: lensed arcs, the lens galaxy,
galaxies along the line of sight, and noise.

Regarding the noise and PSF for the mock HST images, we extracted the
images on the line sight from Hubble Ultra Deep Field, so PSF and
noise are incorporated in the field of view of the mock HST images.

{\bf 
Note that I agree with the further analysis and conclusions that measuring
the wrong abundance will bias the temperature and density profiles in such
a way as to produce systematically wrong results in the mass and gas
fraction in the outskirts. However I am not convinced that this
mis-measurement is always present in CCD spectral analysis of the typical
cluster outskirts ICM, or that it is always mis-measured in the same sense
(i.e. abundance is always underestimated for moderate metal abundance at $\sim
1$~keV).
}

We have explained the systematic mis-measurement of the abundance as a
result of a higher-temperature component in the multi-temperature gas in
the outskirts.  We agree that the abundance will not necessarily be
biased in the same direction for a given low temperature and moderate
metal abudnace, but in our cluster's outskirts, it is the presence of
a higher temperature phase that results in a systematic
under-measurement.  This point is now reflected in the text. 

{\bf 
Section 2.2.1, paragraph 1
``$3\times  R_{\rm vir}$'' $\rightarrow R_{200c}$ is used elsewhere in the manuscript so should be used
here since it is a quantitative value.
}
The projection depth is now also given in terms of ${R}_{200c}$

{\bf 
Section 2.2.2, paragraph 3
``limited flux of $\sim 3 \times 10^{-15}$~erg s$^{-1}$ cm$^{-3}$'' should be ``cm$^{-2}$'' if this
is indeed flux.
}
This is indeed flux - the units are now corrected.

{\bf 
``where we put our clusters at'' $\rightarrow$ delete ``at''
}
The clause has been reworded.
{\bf 
Section 2.2.2 (general)
please specify the abundance table used (e.g. Anders and Grevesse)
}
Yes, for both generating the flux maps and in spectral fitting. This
information has also been added.

{\bf 
Figure 1
What do the green annular sections correspond to? If nothing, they should
be removed. Also, The very small green circles projected over the bright
cluster are a bit hard to see. Finally, please comment on why there are
some blobs not rejected (e.g. just above CL6 in the x and y views, and the
extended filament). Are they just below the threshold for removal?
}

I have added information in the caption, and also the discussion of
Figure 1.  The green annular sections correspond to identified
filamentary regions with X-ray surface brightness above the average in
the whole annular bin (as described in 2.2.2, also reworded for clarity).


{\bf 
Figure 2
This figure is not reference in the text and should be somewhere in Section
2.2.3. Also, since this is a quantity calculated in bins, it would be
helpful to show uncertainties so that the importance of removing these
substructures can be judged. In the caption please specify which cluster
this corresponds to.
}

Figure has been redone with uncertainties in the ratio, and is
discussed in 2.2.3.

{\bf 
Figure 3
Caption is confusing; suggest changing the first line to ``(Left) The
projected X-ray temperature from spectral fitting and (right) the
corresponding projected 3-D density profile for CL7.'' Then lower down delete
``Right: The recovered gas density profile.''
}

Modified.

{\bf 
Figure 4
This figure has a number of issues:
1) x error bars are overlapping, so they are either drawn twice as large as
they should be or for some reason the spectral bins are not independent.
If the latter, please explain why.
2) The residual panel should be plotted as datapoints with error bars, not
solid lines. As shown, it is impossible to judge whether the residuals are
statistically significant.
3) Fitted $T_X$ values need uncertainties quoted, also to show whether the
difference is statistically significant.
4) The y-axis values are hard to believe given the units; this would imply
that there are ~ 10 billion counts in the full 2.4 Msec spectrum in just this
annulus beyond $R_{200c}$, far more than a typical (real) observation. This
is symptomatic of my concern in major issue 2(c) above, so further
clarification of how this mock observation is performed is needed.
}

I've corrected the x-errorbars, added uncertainties to the best fit
X-ray temperature, and modified the axes from $\mathrm{s^{-1}keV^{-1}}$ to
$\mathrm{keV^{-1}}$.  

{\bf 
It would be illuminating to non-expert readers to show a high-resolution
spectrum of a mekal emission model at 1~keV and (e.g.) 6~keV, to show (a)
the complications in the lines in this energy range that are smeared out by
the CCD resolution, and (b) how this really matters for cluster outskirts
where kT is fairly low. This is mentioned elsewhere in the manuscript, but
a picture is worth a thousand words.
}

We have referred the reader to both Bohringer and Werner 2010 and
Mazzotta et al, 2004 for figures with spectral models.  

{\bf 
Figure 5
Upper right panel: is the ``$f_{T_{X,{\rm proj}}}$'' the same sort of ``fit/fix'' mentioned
in the caption? The text in Section 3.2, paragraph 5 suggests this (also
`note the runaway parenthesis there). This needs to be clarified, perhaps
with a different label including ``fit/fix'' explicitly or some of the text
from that paragraph placed in the caption.
Lower left panel: Error bars would be very useful to judge whether the
differences are significant.
}

Errorbars have been added and the caption is clarified.

{\bf 
Section 3.2, paragraph 7 and bullet point 3 in the summary
``temperature profile...does not accurately describe the thermal
behavior at larger radii.''
I disagree; as shown in Fig 5 upper right, the data points are almost fully
consistent with the model profiles within the error bars. Again, error
bars in the lower plot would qualitatively show this.
}

While consistent with the errorbars, the shape of the best-fit
projected X-ray temperature profile does not capture the emerging
trend of the decreasing slope in the projected X-ray data.  We
interpret this as another indication of the complicated temperature
structure in cluster outskirts and have commented on this in the
section.

{\bf 
Section 3.2, paragraph 8
``... to have the statistical uncertainty...''
$\rightarrow$ ``...to have the same statistical uncertainty...''
}

Modified.

{\bf 
Section 3.2, paragraphs 9 and 11 (last)
Variations in a number of profiles are quoted, but how do these values
compare to the propagated statistical error of the measurements or inferred
values? i.e., are they significant variations?
}

I have added errorbars to the variations in X-ray temperature and
binned EMM.

{\bf 
Figure 6
This figure has the same issues as Figure 4, and they should be addressed.
}

I have made similar changes as figure 4, but adding errorbars to the
residuals for the data makes the plot rather overloaded.  As the residuals
are the difference between the mock photon spectrum and each of the
model spectra, the errors in the upper panel directly translate to
what would be in the lower panel.  

{\bf 
Section 3.3, last paragraph
The effects of extrapolating the abundance to radii where it can't be
measured is a key point, because (1) essentially all Suzaku observational
studies do this, and (2) it *will* cause a bias notwithstanding my concerns
in major issue 1. I think this should be stressed more in the summary,
perhaps as a bullet point.
}

This effect has now been emphasized in the abstract, section 3.3, and
in the summary bullet points.

{\bf 
Section 3.4, paragraph 4
Should this be an abundance ``5 times lower", not "higher''? That is what
Figure 9 right suggests.
}

Yes, modified.

{\bf 
Section 3.4, paragraph 5
``...where photon statistics are poor.''Also at small radii the temperature
is quite a bit hotter, and the metal abundance becomes more separated from
temperature in the spectral fit. This should be mentioned.
}

Added.

{\bf 
Section 3.5, paragraph 3
``...but different...'' $\rightarrow$ ``...but differ...''
}

Modified.

{\bf 
Section 4, last paragraph
Footnotes for SMART-X and Athena don't appear.
}

These appear in our compilations of the manuscript, but we will ask
the journal editor to help make sure these appear in the ApJ compiled version.

\end{document}
